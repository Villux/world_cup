Modeling football scores has gained some interest in scientific communities. One of the first to mention that football score distribution resembles a Poisson distribution was Maroney \cite{moroney1962facts} in his book 'Facts from figures'. Although the Poisson distribution fits the scores well, Maroney came into a conclusion that the negative binomial distribution fits the scores better. But was this enough? In 1997 Dixon stated \cite{dixon1997} "It is not difficult to predict fairly accurately which teams are likely to be successful, but the development of models that have a sufficiently high resolution to exploit this long run predictive capability for individual matches is substantially more difficult." Chance dominates the game in football \cite{ben2006parity}. However, Maher's \cite{maher1982modelling} idea to include the team's quality into the model was promising. He used an independent Poisson distribution to model the score based on the team's previous performance. Ten years later Dixon used a similar approach in his model and was able to achieve positive returns from betting \cite{dixon1997}.

To differentiate the teams the key is to describe the team's abilities as well as possible. Elo rating, initially developed for assessing the strength of chess players \cite{elo1978rating}, has gained popularity in football prediction. Leitner et al. \cite{leitner2010forecasting} simulated UEFA Euro 2008 football tournament using ratings of abilities such as the Elo rating and the FIFA/Coca-Cola World ranking with bookmaker's odds. They showed that Elo and FIFA/Coca-Cola World ranking are suitable attributes for match outcome prediction, but not as good as the bookmaker's odds are. The FIFA/Coca-Cola World\footnote{FIFA has changed the ranking algorithm three times since it was introduced. The ranking algorithm will be modified after FIFA World Cup 2018 to resemble Elo rating \cite{wiki:fifarating}.} rating had a higher Spearman correlation with the tournament outcome than the Elo rating. This means that it would be a better metric if the corresponding winning probabilities could be computed. \cite{leitner2010forecasting} The results of Lasek et al. \cite{lasek2013predictive} are contrary to this finding. When they compared rating methods, the Elo rating described the team's ability better than the FIFA/Coca-Cola rating. Hvattum et al. \cite{hvattum2010using} concluded that the single rating difference is a highly significant predictor of the match outcomes, which justifies the increasing interest in using Elo rating to describe the team's ability. In their study, they used logistic regression but were not able to achieve as good results with Elo rating as they were with market odds. More data is needed with Elo rating to match the market's accuracy.

When the number of used features has grown, tree-based models have gained popularity in football prediction. Groll et al.\cite{groll2018prediction} predicted the FIFA World Cup 2018 match outcomes using three models: a random forest model, a regression model, and a ranking model. From these models, the random forest model performed the best in classification and even outperformed the bookmakers. For features, they used economic factors such as GDP per capita, sportive factors like FIFA/Coca-Cola rating, factors that described the team's structure such as average age and factors that described the team's coach. In another study, where the outcome of a match in the Turkish super league was predicted, a random forest model was the best performing model. It outperformed a support vector machine model and a bagging REP tree model in prediction accuracy. In this study feature set selection was performed and a limited feature set performed better than the full feature set. \cite{10.1007/978-3-319-29504-6_48}

One clear motivation behind the match outcome prediction is the possibility to earn profit from betting. If the markets are efficient, an investor cannot consistently "beat the market" \cite{badarinathi1996football}. This assumption is known as the \textit{Efficient market hypothesis} (EMH) and it comes from the field of economics. Since American football's betting market resembles Wall Street, Pankoff \cite{pankoff1968market} reasoned that American football betting should satisfy the EMH as well. He concluded that the systematic market errors are not large enough to be profitable to bettors. Later the efficient market hypothesis has been revoked in finance \cite{jegadeesh1993returns}, and this might also be the case in betting \cite{goddard2003modelling, badarinathi1996football}. However, Goddard et al. \cite{goddard2003modelling} state that there is some evidence that the inefficiencies in the bookmakers’ prices have diminished over time. Kuypers \cite{kuypers2008} analysis on how bookmakers calculate their odds supports this claim of price inefficiency. If bookmakers want to increase their profits while keeping their over-roundness competitive, they need to set the odds further from the market efficient odds. Also, marketing purposes and heavy one-sided bets can lead to inefficient odds. In the case of heavy one-sided bets, bookmakers expose themselves to a higher risk exposure if the match outcome is the one that is betted very heavily. With a proper betting strategy and a model that can beat bookmakers generating profit from football betting should be possible.

One well-known betting strategy is Kelly's criterion which has also been used outside of sports betting by famous investors like Warren Buffet and Bill Cross\footnote{http://www.financial-math.org/blog/2013/10/two-tales-of-the-kelly-formula/}. This strategy aims to maximize the logarithm of wealth by placing an optimal fraction of the total bankroll for a bet based on the probabilities \cite{kelly2011new}. MacLean et al.\cite{maclean1992growth} looked at the benefits of using a fraction of the optimal fraction given by the Kelly criterion. They concluded that there is a tradeoff between a small decrease in the growth rate against the increased chance of doubling your fortune before it is halved. This is a crucial finding since World Cup tournaments have a very limited number of games and for a strategy to work, it needs to be successful from the very first games onwards.

Data collection around football has increased. Betting agencies require more data to improve their prediction accuracy and many professional bettors, football scouts, and coaches are also interested in this data. More expressive datasets are valuable and hard to collect which makes it challenging to find a clean and a comprehensive dataset for free. One interesting source for open data is video games. To simulate the game accurately a considerable amount of data is required. EA Sport's video game series FIFA has included a comprehensive player attribute dataset from the year 2006 onwards. Shin et al.\cite{shin2014novel} investigated this dataset and showed that it describes the players well. When they compressed the data into 3D space, they were able to see clear separations between attackers, defensive players, and goalkeepers. With this "virtual data" (data from the video game) they were able to make more accurate predictions compared to predictions that used "real data" , which contained different statistics from historical matches.
