Modeling football scores has gained some interest in scientific communities. One of the first to mention that football scores can resemble Poisson distribution was Maronoey if his book Facts from figures. Although Poisson distributed fitted the scores well he came to a conclusion that using the negative binomial distribution could be an improvement.\cite{moroney1962facts}
In 1997 Dixon stated \cite{dixon1997} "It's not difficult to predict fairly accurately which teams are likely to be successful, but the development of models that have a sufficiently high resolution to exploit this long run predictive capability for individual matches is substantially more difficult." To his knowledge the only one who had included team's quality into the model was Maher. He used independent Poisson distribution is his model to model the score based on team's previous performance. \cite{maher1982modelling} Over ten years later Dixon used a similar approach in his model and was able to achieve positive returns from betting \cite{dixon1997}.

The key with these kind of models is to describe the team's quality as well as possible. Elo rating, originally developed for assessing the strength of chess players \cite{elo1978rating}, has gained popularity in football prediction. Leitner et al. \cite{leitner2010forecasting} simulated UEFA Euro 2008 football tournament using ratings of abilities (such as the Elo rating or FIFA/Coca-Cola World ranking) and Bookmaker's odds. They showed that Bookmarker's odds outperformed Elo, but both are suitable attributes for match outcome prediction. One interesting finding in \cite{leitner2010forecasting} is that FIFA/Coca-Cola World \footnote{FIFA has changed the ranking algorithm three times since it was introduced. The ranking algorithm will be changed after FIFA World Cup 2018 to resemble Elo rating \cite{wiki:fifarating}.} rating has a higher Spearman correlation than Elo rating and could be a good metric if the corresponding winning probabilities could be computed. Results in paper \cite{lasek2013predictive} are contrary to these findings. Lasek et al. \cite{lasek2013predictive} compared different rating methods and their results imply that Elo rating describes a better team's ability compared to FIFA/Coca-Cola rating. Hvattum et al. \cite{hvattum2010using} concluded that the single rating difference is a highly significant predictor of match outcomes which justifies the increasing interest in using Elo rating to describe team's ability. In their study they used logistic regression, but were not able to achieve as good results with using Elo rating only as they were with market odds. More data is needed with Elo rating to match the market's accuracy.

Groll et al. \cite{groll2018prediction} predicted the FIFA World Cup 2018 match outcomes using three different methods. Compared to many previous studies their dataset was more comprehensive. The dataset was the same as it was in their previous study \cite{groll2015prediction}. They used economic factors such as GDP per capita, sportative factors like FIFA rank, factors that described the team's structure like average age and factors that described team's coach. From random forest model, regression model and ranking model the random forest model performed the best in classification and outperformed the bookmakers.

One clear motivation behind match outcome prediction is the possibility to earn money from betting. The \textit{Efficient market hypothesis} (EMH) defined by Badarinathi and Kochmann \cite{badarinathi1996football} "asserts that investors cannot consistently "beat the market" because stocks reside in perpetual equilibrium" is a known assumption in finance. Weak-form efficiency means that a part of the odds given by bookmakers are mispriced. The efficient market hypothesis has been revoked in finance \cite{jegadeesh1993returns}. Goddard et al. and Badarinathi et al. \cite{goddard2003modelling, badarinathi1996football} find that this might be the case in also in betting. However Goddard et al. \cite{goddard2003modelling} states that there is some evidence that inefficiencies in the bookmakers’ prices have diminished over time. Kuypers \cite{kuypers2008} analysis on how bookmakers calculate their odds supports the claim of inefficiencies in the bookmakers’ prices. If bookmaker wants to increase its profits while keeping its over-roundness competitive it needs set the odds further from the market efficient odds. Also, marketing purposes and heavy one-sided bets can lead to inefficient odds. In the case of heavy one-sided bets bookmaker exposes itself to higher risk exposure if match's outcome is the one that is betted very heavily. With a proper betting strategy and a model that is able to beat bookmakers generating profit from football betting should be possible.

One well-known betting strategy is Kelly's criterion which has been also used outside of sport betting by famous investors like Warren Buffet and Bill Cross \footnote{http://www.financial-math.org/blog/2013/10/two-tales-of-the-kelly-formula/}. This strategy aims to maximize the logarithm of wealth by placing an optimal fraction of the total bankroll for a bet based on the probabilities. MacLean et al. \cite{maclean1992growth} looked at the benefits of using a fraction of the optimal fraction given by Kelly criterion. They concluded that it is a trade off between small decrease in the growth rate against increased change of doubling your fortune before it's halved. This is a crucial finding since World Cup tournaments have very limited number of games and for a strategy to work, it needs to be successful from the very first games onwards.


% \begin{enumerate}
%     \item What methods and how matches are predicted (Win/lose, score)
%     One factor models, Multifactor models, Score vs outcome prediction, data used in multifactor,
%     \item How simulations are done if any
%     \item Elo rating
%     \item beating bookmaker's odds
% \end{enumerate}



% Groll et al. \cite{groll2018prediction} predicted the FIFA World Cup 2018 match outcomes using three different methods. Data used in their experiments was the as in \cite{groll2015prediction}. This dataset has very limited attributes to describe teams playing styles. Team's ability covariate from the rating method is included in the input vector for the final simulations. Team's ability weights team's recent performance more than FIFA/Coca-Cola World ranking . Including this team's ability parameter improved the accuracy significantly for Random Forest. It's important to notice that the input data used in \cite{groll2018prediction} is not updated during the tournament. This means that ranking stays static even though one team might perform well based on the other attributes and survive far in the tournament. This means that the team's ability is not updated to reflect the real ability based on the prediction results.
% Groll et al. \cite{groll2018prediction} predicted the match outcomes using regression, not classification. With regression, the feature vector is conditional on match's teams but predicted scores are drawn from independent Poisson distributions. With ranking methods score is dependent. Dixon and Coles \cite{dixon1997} identified correlation between scores and are one of the many researchers that have relaxed the strong assumption of conditional independence between the scores.

% Leitner et al. \cite{leitner2010forecasting} simulated UEFA Euro 2008 football tournament using ratings of abilities (such as the Elo rating or FIFA/Coca-Cola World ranking) and Bookmaker's odds. They showed that Bookmarker's odds outperformed Elo, but both are suitable attributes for match outcome prediction. Their models predicted only the probability of a win. This limits the model's accuracy in cases where two or more teams share the same number of points in the group stage. The model which based on the bookmaker's odds was able to predict the teams in the final correctly. One interesting finding in \cite{leitner2010forecasting} is that FIFA/Coca-Cola World \footnote{FIFA has changed the ranking algorithm three times since it was introduced. The ranking algorithm will be changed after FIFA World Cup 2018 to resemble Elo rating \cite{wiki:fifarating}.} rating has a higher Spearman correlation than Elo rating and could be a good metric if the corresponding winning probabilities could be computed. Results in paper \cite{lasek2013predictive} are contrary to these findings. Lasek et al. \cite{lasek2013predictive} compared different rating methods and their results imply that Elo rating describes a better team's ability compared to FIFA/Coca-Cola rating.

% Simulations \cite{leitner2010forecasting, groll2018prediction} are done in a manner which takes into account group draws. Their results show that teams with relatively easier opponents in the group stage have higher probabilities to go further in the tournament.

% One clear motivation behind match outcome prediction is the possibility to earn money from betting. The \textit{Efficient market hypothesis} defined by Badarinathi and Kochmann \cite{badarinathi1996football} "asserts that investors cannot consistently "beat the market" because stocks reside in perpetual equilibrium" is an assumption used in finance. Weak-form efficiency means that a part of the odds given by bookmakers are mispriced. The efficient market hypothesis has been revoked in finance \cite{jegadeesh1993returns}. Goddard et al. and Badarinathi et al. \cite{goddard2003modelling, badarinathi1996football} find that this might be the case in also in betting. However Goddard et al. \cite{goddard2003modelling} states that there is some evidence that inefficiencies in the bookmakers’ prices have diminished over time.

% Kuypers \cite{kuypers2008} shares some light on how bookmakers calculate their odds. If bookmaker wants to increase its profits while keeping its over-roundness competitive it needs set the odds futher from the market effiecient odds. Also, marketing purposes and heavy one-sided bets can lead to inefficient odds. In the case of heavy one-sided bets bookmaker exposes itself to higher risk exposure if match's outcome is the one that is betted very heavily.
