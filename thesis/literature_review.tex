Modeling football scores has gained some interest in scientific communities. One of the first to mention that football score distribution resembles a Poisson distribution was Maroney \cite{moroney1962facts} in his book Facts from figures. Although Poisson distribution fitted the scores well he came into a conclusion that using the negative binomial distribution could be an improvement. But was this enough? In 1997 Dixon stated \cite{dixon1997} "It's not difficult to predict fairly accurately which teams are likely to be successful, but the development of models that have a sufficiently high resolution to exploit this long run predictive capability for individual matches is substantially more difficult." Chance dominates the game. However, Maher \cite{maher1982modelling} idea to include the team's quality into the model was promising. He used independent Poisson distribution is his model to model the score based on the team's previous performance. Over ten years later Dixon used a similar approach in his model and was able to achieve positive returns from betting \cite{dixon1997}.

To differentiate the teams the key is to describe the team's abilities as well as possible. Elo rating, originally developed for assessing the strength of chess players \cite{elo1978rating}, has gained popularity in football prediction. Leitner et al. \cite{leitner2010forecasting} simulated UEFA Euro 2008 football tournament using ratings of abilities such as the Elo rating and FIFA/Coca-Cola World ranking with Bookmaker's odds. They showed that Bookmarker's odds outperformed Elo and FIFA/Coca-Cola World ranking, but both are suitable attributes for match outcome prediction. When ratings of abilities were compared FIFA/Coca-Cola World \footnote{FIFA has changed the ranking algorithm three times since it was introduced. The ranking algorithm will be changed after FIFA World Cup 2018 to resemble Elo rating \cite{wiki:fifarating}.} rating had a higher Spearman correlation than Elo rating which means that it would be a better metric if the corresponding winning probabilities could be computed \cite{leitner2010forecasting}. Lasek et al. \cite{lasek2013predictive} results are contrary to this finding. They compared different rating methods and their results imply that Elo rating describes better team's ability compared to FIFA/Coca-Cola rating. Hvattum et al. \cite{hvattum2010using} concluded that the single rating difference is a highly significant predictor of match outcomes which justifies the increasing interest in using Elo rating to describe the team's ability. In their study, they used logistic regression but were not able to achieve as good results with Elo rating as they were with market odds. More data is needed with Elo rating to match the market's accuracy.

When the number of used features has grown tree-based models have gained popularity in football prediction. Groll et al. \cite{groll2018prediction} predicted the FIFA World Cup 2018 match outcomes using three models: a random forest model, a regression model, and a ranking model. From these models, the random forest model performed the best in classification and even outperformed the bookmakers. For features, they used economic factors such as GDP per capita, sportive factors like FIFA/Coca-Cola rating, factors that described the team's structure like average age and factors that described the team's coach. In another study where the outcome of a match in the Turkish super league was predicted a random forest model was the best performing model. It outperformed a support vector machine model and a bagging REP tree model in prediction accuracy. In this study feature set selection was performed and a limited feature set performed better than the full feature set. \cite{10.1007/978-3-319-29504-6_48}

One clear motivation behind match outcome prediction is the possibility to earn money from betting. The \textit{Efficient market hypothesis} (EMH) defined by Badarinathi and Kochmann \cite{badarinathi1996football} "asserts that investors cannot consistently "beat the market" because stocks reside in perpetual equilibrium" is a known assumption in finance. Since american football betting market resembles Wall Street Pankoff \cite{pankoff1968market} reasoned that american football betting should be EMH as well. He came to a conclusion that the systematic market errors are not large enough to be profitable to bettors. Later the efficient market hypothesis has been revoked in finance \cite{jegadeesh1993returns}, and this might be the case in betting as well \cite{goddard2003modelling, badarinathi1996football}. However Goddard et al. \cite{goddard2003modelling} states that there is some evidence that inefficiencies in the bookmakers’ prices have diminished over time. Kuypers \cite{kuypers2008} analysis on how bookmakers calculate their odds supports this claim of inefficiencies in the bookmakers’ prices. If bookmaker wants to increase its profits while keeping its over-roundness competitive it needs set the odds further from the market efficient odds. Also, marketing purposes and heavy one-sided bets can lead to inefficient odds. In the case of heavy one-sided bets, bookmaker exposes itself to higher risk exposure if the match's outcome is the one that is betted very heavily. With a proper betting strategy and a model that is able to beat bookmakers generating profit from football betting should be possible.

One well-known betting strategy is Kelly's criterion which has been also used outside of sports betting by famous investors like Warren Buffet and Bill Cross \footnote{http://www.financial-math.org/blog/2013/10/two-tales-of-the-kelly-formula/}. This strategy aims to maximize the logarithm of wealth by placing an optimal fraction of the total bankroll for a bet based on the probabilities. MacLean et al. \cite{maclean1992growth} looked at the benefits of using a fraction of the optimal fraction given by Kelly criterion. They concluded that it's a tradeoff between a small decrease in the growth rate against increased change of doubling your fortune before it's halved. This is a crucial finding since World Cup tournaments have a very limited number of games and for a strategy to work, it needs to be successful from the very first games onwards.

Data collection around football has increased. Betting agencies require more data to improve their prediction accuracy and many professional bettors, football scouts and coaches are also interested in this data. More expressive datasets are valuable and hard to collect which makes finding a clean and comprehensive dataset for free hard. One interesting place to look for free data are video games. To simulate the game properly a lot of data is required. EA Sport's video game series FIFA has included comprehensive player attribute dataset from the year 2007 onwards. Shin et al. \cite{shin2014novel} investigated this dataset and showed that it can describe the players well. When they compressed the data into 3D space they were able to see clear separations between attackers, defensive players, and goalkeepers. With this "virtual data" (data from the video game) they were able to predict more accurately than with "real data" which contained different statistics from historical matches.

