In this thesis I have investigated the possibility to "beat the bookie" in prediction accuracy and in betting on FIFA World Cup. In my experiments I have used four different models and different combinations of features. For betting two different strategies were used. All the data used in the experiments was freely available on the internet.

Reference model, the bookmaker's model, was formed using the average odds provided for the match. This model achieved the accuracy of 56.25\% for World Cup 2018 and 2014, and 51.56\% for World Cup 2010. In total, 28 out of the all 36 model and feature set combinations where able to beat the bookmaker's model in prediction accuracy. "Beating the bookie" in prediction accuracy is doable.

Being profitable with both of the betting strategies in all of the tested World Cup tournaments: 2018, 2014, and 2010 is possible. More importantly it's possible to earn on average as high as 24.05\% returns using Kelly strategy. Unit strategy is a useful and has less variation, but it can't utilize the bank as well as Kelly and is not able to maximize the profits from abnormal odds. Yes, "Beating the bookie" in profitability is also doable. However, this result should be approached with caution. Combined results from three World Cups contains only 192 games, which means that the sample is small. The lag of 4 years between the tournaments gives bookmakers plenty of time to improve their models. Already results from the latest World Cup indicate that good opportunities exists more rarely; profits are more connected to few games and only 7 out of 12 of the models were able the more accurate than the reference was. There are no guarantees that the profits from the upcoming World Cup 2022 will be positive.

Extensive feature analysis with tree-based models indicate that no subset of features gives better results than using all of the features. Some of the tournament simulation results are in contrary to this since in some cases using a limited feature set improves the single tournament simulation results. However, when all tournaments are considered using all features seems to be the best option. Randomness in football could be the reason why some of the tournaments perform better with limited feature sets. Instead of optimizing the perfect feature set the features themselves should be investigated. Maybe the team-level aggregating process is not the best to really differentiate the teams. Also, more data could be included to improve the feature accuracy. Using exact lineups for every game when team-level attributes are created is a promising way to improve the features. Using optimal hyperparameters turned out to be an important way to improve the results. The grid search strategy for finding the optimal hyperparameters was successful.

Predicting draw turned out to be really demanding. What could be done to improve this? This thesis will not provide a clear answer. Based on the results chance dominates the game so much that predicting draws more accurately is demanding. Models predicted draws differently but no model was clearly better than the rest. The reference model was no better in accuracy.

