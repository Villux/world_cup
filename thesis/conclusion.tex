In this thesis, we have investigated the possibility to "beat the bookie" on FIFA World Cup tournaments. To achieve this, the profits from betting need to be positive. In our experiments, we have used four different models with three different combinations of features. For betting, two different strategies were used. All the data used in the experiments was freely available on the internet.

Reference model, the bookmaker's model, was formed using the average odds provided for the match. This model achieved the accuracy of 56.25\% for World Cup 2018 and 2014, and 51.56\% for World Cup 2010. In total, 28 out of the all 36 model and feature set combinations were able to beat the bookmaker's model in prediction accuracy. "Beating the bookie" in prediction accuracy is doable.

Being profitable with both of the betting strategies in all of the tested World Cup tournaments: 2018, 2014, and 2010 is possible. More importantly, it's possible to earn on average as high as 24.05\% returns using the Kelly strategy. The other strategy, the unit strategy, has less variance but the fixed bet size limits its ability to maximize the profits from favorable odds. Unit strategy's returns were more often positive, but lower in size. The answer to the question
"Is it possible to 'beat the bookie'?" is \textit{yes}. However, this result should be approached with caution. Combined results from three World Cups contains only 192 games, which means that the sample is small. It would be ideal to test the models with more tournaments if data would be available.
Also, the lag of 4 years between the tournaments gives bookmakers plenty of time to improve their models. Already results from the latest World Cup indicate that excellent opportunities exist more infrequently; profits are more connected to few games, and only 7 out of the 12 model and feature set combinations were able to outperform the reference model's accuracy. There are no guarantees that the profits from the upcoming World Cup 2022 will be positive. In the future work, these methods could be used to simulate games in different football leagues. More games can be used for validation which gives more confidence in the results.

Extensive feature analysis with tree-based models indicated that no subset of features gave better results than using all of the features. Some of the tournament simulation results were in contrary to this since in some cases using a limited feature set improved the result of a single tournament simulation. However, when all tournaments were considered, using all of the features seems to be the best option. Instead of further optimizing the perfect feature set, the features themselves should be investigated. Maybe the current way of aggregate the team-level features could be improved to differentiate the teams better. Also, more data, like lineups, could be included to enhance the feature's accuracy. Using optimal hyperparameters turned out to be an essential way to improve the results. The grid search strategy for finding the optimal hyperparameters was successful.

Predicting draws turned out to be demanding. What could be done to improve this? This thesis will not provide a clear answer. Models predicted draws differently, but no model was clearly better than the rest. The reference model was no better in accuracy. Improving the prediction accuracy of a draw would be most likely very beneficial and hopefully initiates further research.
