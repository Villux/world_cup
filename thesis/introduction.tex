\section{Objectives of the Thesis}
\section{Literature Review}
Groll et al. \cite{groll2018prediction} predicted the FIFA World Cup 2018 match outcomes using three different methods. They used the same data as in \cite{groll2015prediction} in the first part and included team's ability covariate from rating method for the input covariates that is used in the final simulations. Including this team's ability parameter improved the accuracy significantly for Random Forest. It's important to notice that input data used in \cite{groll2018prediction} is not updated during the tournament. This means that ranking stays static even though one team might perform well based on the other attributes and survive far in the tournament. This means that team's ability is not updated to reflect the real ability based on the prediciton results.
Groll et al. \cite{groll2018prediction} predicted the match outcomes using regression, not classification. With regression feature vector is conditional on match's teams but predicted scores are drawn from independent Poisson distributions. With ranking methods score are dependant. Dixon and Coles \cite{dixon1997} identified correlation between scores and are one the many researches that have relaxed the strong assumption of conditional independence between the scores.
