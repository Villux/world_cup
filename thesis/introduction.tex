The main motivation behind this master's thesis is to experiment how accurately an outcome of a football match in a World cup tournament can be predicted and is it possible to create a profitable betting strategy for the tournament. To "beat the bookie" only freely accessible data can be used.

Football, being one of the biggest sports in the world, attracts a lot of global attention especially during the main tournaments like the FIFA World Cup and EURO. This global interest has attracted big investment banks like Goldman Sachs and UBS to predict tournament outcome. Also, the research community has been active around this topic; mostly predicting the probabilities for winning the tournament or predicting the most probable course of the tournament \cite{groll2018prediction, groll2015prediction, leitner2010forecasting}.

For many betting the outcome of a football match is as important part of the experience as watching the game. No wonder that the size of the betting industry is massive. This must be one the reason why the field of economics has actively research this industry, especially the efficiency of the betting market. When markets are efficient all the available information is reflected in the prices and there is no change to "beat the market". This efficient market hypothesis of the betting market has been discarded by \cite{vlastakis2009efficient} - even the weak-form of the efficient market hypothesis. Even though the betting markets are not efficient the general assumption is that probabilities given by the betting industry are close to the true probabilities. For this reason, models are often benchmarked against the odds given by the betting industry and can be useful features as well. \cite{leitner2010forecasting}. Kuypers \cite{kuypers2008} has heavily claims against the belief that the betting market consensus is close to the true probabilities. If bookmaker wants to increase its profits while keeping its over-roundness competitive it needs set the odds further from the market efficient odds. Also, marketing purposes and heavy one-sided bets can lead to inefficient odds. In the case of heavy one-sided bets, bookmaker exposes itself to a higher risk exposure if the match's outcome is the one that is betted very heavily. Major tournaments like FIFA World Cup increase the betting activity. The hypothesis is that this can lead to odds that are further away from the true probabilities which naturally leads to a question: is it possible to earn profit from betting FIFA World Cup matches?

As far as I know, this opportunity to bet profitably on FIFA World Cup has not been researched. For this reason, one of the main goals of this master's thesis is to investigate if it's possible to create a profitable betting strategy for the FIFA World Cup.

Nowadays, data is collected extensively from different sports. For long data collection in football was limited compared to other sports like basketball, but now many different sources for data exists. The unfortunate part is, that the data is very expensive and only part of it is freely available. Free data mostly contains historical matches. Fortunately many games that simulate football have a vast collection of player and team-level attribute data. This data is freely available and to my knowledge has been successfully used once \cite{shin2014novel}. Can this freely available data on players and historical matches be used to predict the tournaments more accurately than the betting market predicts?

This thesis will answer to these two questions: "Is it possible to bet FIFA World Cup matches profitably?" and "Is it possible to predict the tournaments more accurately than the betting market predicts?". The idea is not to seek for abnormal odds but to use the average odds from the markets wisely. If positive profits exist in this scenario, "beating the bookie" is not impossible.

The rest of the thesis is structured as follows: in Literature review, I go through the existing research on the topic. Next, in Section 3, I describe the dataset and how player attributes are aggregated to team-level attributes. Then, in Section 4, I describe the prediction models, betting strategies and the tournament simulation process. In Section 5 results from the simulation experiments are stated. Finally, I conclude and discuss future work in Section 6.