\section{Objectives of the Thesis}
\section{Literature Review}

\begin{enumerate}
    \item What methods and how matches are predicted (Win/lose, score)
    \item How simulations are done if any
    \item Elo rating
    \item beating bookmaker's odds
\end{enumerate}

Groll et al. \cite{groll2018prediction} predicted the FIFA World Cup 2018 match outcomes using three different methods. Data used in their experiments was the as in \cite{groll2015prediction}. This dataset has very limited attributes to discribe teams playing styles. Team's ability covariate from rating method is included in the input vector for the final simulations. Team's ability weights team's resent performance more than FIFA/Coca Cola World ranking . Including this team's ability parameter improved the accuracy significantly for Random Forest. It's important to notice that input data used in \cite{groll2018prediction} is not updated during the tournament. This means that ranking stays static even though one team might perform well based on the other attributes and survive far in the tournament. This means that team's ability is not updated to reflect the real ability based on the prediciton results.
Groll et al. \cite{groll2018prediction} predicted the match outcomes using regression, not classification. With regression feature vector is conditional on match's teams but predicted scores are drawn from independent Poisson distributions. With ranking methods score are dependant. Dixon and Coles \cite{dixon1997} identified correlation between scores and are one the many researches that have relaxed the strong assumption of conditional independence between the scores.

Leitner et al. \cite{leitner2010forecasting} simulated UEFA Euro 2008 football tournament using ratings of abilities (such as the Elo rating or FIFA/Coca-Cola World ranking) and Bookmaker's odds. They showed that Bookmarker's odds outperformed Elo, but both are suitable attributes for match outcome prediction. Their models predited only the probability of a win. This limits the models accuracy in cases where two or more teams share the same number of points in group stage. Model which based on bookmaker's odds was able to predict the teams in the final correctly. One interesting finding in \cite{leitner2010forecasting} is that FIFA/Coca Cola World \footnote{FIFA has changed ranking algorithm three times since it was introduced. Ranking algorithm will be changed after FIFA World Cup 2018 to resemble Elo rating \cite{wiki:fifarating}.} rating has a higher Spearman correlation than Elo rating and could be a good metric if the corresponding winning probabilities could be computed. Simulations are done in a manner which takes to account group draws. Their results show that teams with relatively easier opponents in the group stage have higher probabilities to go further in the tournament.
