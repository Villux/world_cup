\section{Predicting football match outcome with Random Forest Classifier}
What is is done. Why random forest? What are the alternative methods
\subsection{Random Forest}
\label{ss:randomforest}
In machine learning supervised learning is the task of learning the relationship between input features and and the target value. Structure that describes this relathionship is called a model. In most cases these models are used to predict the target value based on new input features. There are two types of models: \textit{regression models} and \textit{classifier models}. If the target value is in a real-valued domain the model is called \textit{regression model}. \textit{Classifier models} are used to map the input features to predefined classes. \cite{rokach2005top}

Decision tree is one of the most popular model type used in classification problems. Decision tree is a directed tree, which means that all of the nodes, except the \textit{root node}, have exactly one incoming edge. Nodes that have outgoing edges are called \textit{internal nodes} and the nodes that have only incoming edge are called the \textit{leaf nodes}. Internal nodes in decision tree split the instance space into two or more subspaces according to a certain discrete function of the input's feature values. Usually split is done based on a single feature from the whole feature vector. A single class value is assigned for the \textit{leaf nodes}. When new input is given tree is navigated from the \textit{root node} to a \textit{leave node} which determinates the predicted class label. In regression these target values can take continuous values. \cite{rokach2005top}

Decision trees have many benefits and are very useful "off-the-shelf" predictors. Outliers in the dataset or many irrelevant predictors are not problematic for the trees. Scaling or any other general transformation can be done to the input space since trees are invariant under transformation of the individual predictors. \cite{friedman2001elements} Decision trees have good interpretability if the trees are small.

On main disadvantage of the decision trees is bad prediction accuracy \cite{friedman2001elements}. Decision trees can express the training data well, but have high variance which means that prediction accuracy for unseen data is often worse compared to other models.

Bootstrap aggragating, also called bagging, is a way to improve the prediction accuracy of decision trees by averaging. In bagging the average is taken over the output of a multiple estimators:
\begin{align}
    \hat {f_{bag}}(x) = \frac{1}{B}\sum_{b = 1}^{B} \hat {f}^{*b}(x)
\end{align}
where B is the number of estimators and ${f}^{*}$ is a single estimator. This reduces the high variance of a single tree and makes predictions more accurate.

Bootstrap in bagging means that in the training of a single tree a random sample with replacement is taken from the original sample. Samples used in training come from the same distribution, meaning that the trees are identically distributed (i.d.). This combined with deep trees that have less bias ensures that the variance reduction achieved in bagging comes with a expense of small increase in bias and the loss of interpretability. The loss of interpretability can't be avoided since a single tree can't be used anymore for reasoning. Trees in bagging are only identically distributed. The missing independent property means that the trees in the forest can have pairwise correlation. This is common in cases where input data has one strong predictor which often leads to a situation where all of the tree are splitted similarily. \cite{friedman2001elements}

Amit and Geman's \cite{amit1997shape} idea of random feature selection inspired Breiman to use bagging in tandem with random feature selection. With this random feature selection correlation between the trees can be reduced since the generalization error of a forest of tree classifiers depends on the strength of the individual trees in the forest and the correlation between them \cite{breiman2001random}. Breiman was first to use the name \textit{Random Forest} for algorithms that use bagging and random feature selection with tree predictors \cite{breiman2001random}. Step-by-step instruction from \cite{friedman2001elements} for Random Forest algorithm are listed in the Algorithm \ref{alg:random_forest}.

Main usecases for Random forest are \textit{classification} and \textit{regression}.

\begin{algorithm}
    \footnotesize
    \begin{minipage}{.92\linewidth}
    \begin{enumerate}
        \item For $b = 1$ to $B$:
        \begin{enumerate}
            \item Draw a bootstrap sample $\bm{Z}^{*}$ of size $N$ from the training data.
            \item Grow a random-forest tree $T_b$ to the bootstrapped data, by recursively repeating the following steps for each terminal node of the tree, until the minimum node size $n_{min}$ is reached.
            \begin{enumerate}
                \item Select $m$ variables at random from the $p$ variables.
                \item Pick the best variable/split-point among the $m$.
                \item Split the node into two daughter nodes.
            \end{enumerate}
        \end{enumerate}
        \item Output the ensemble of trees $\left\{ T _ { b } \right\} _ { 1 } ^ { B }$.
    \end{enumerate}
    To make a prediction at a new point $x$:

    \textit{Regression:} $\hat { f } _ { \mathrm { rf } } ^ { B } ( x ) = \frac { 1 } { B } \sum _ { b = 1 } ^ { B } T _ { b } ( x )$

    \textit{Classification:} Let $\hat { C } _ { b } ( x )$ be the class prediction of the $b$th random forest
    tree. Then $\hat{C} _ { r f } ^ { B } ( x )$ = \textit{majority vote} $\left\{ \hat { C } _ { b } ( x ) \right\} _ { 1 } ^ { B }$
    \end{minipage}
    \caption{\footnotesize Random Forest for Regression or Classification.}
    \label{alg:random_forest}
\end{algorithm}

\subsection{Random forest hyperparameter selection}
Many machine learning algorithm have parameters that are not optimized within the algorithm itself. These parameters are called hyperparameters. Optimizing these hyperparameters is one way to improve the models performance, since optimal parameters are often problem specific. Random forest is no exception, even though in many cases its performance is relatively decent with default parameters \cite{probst2018hyperparameters}.

For this problem I decided to optimize three hyperparameters: \textit{number of candidate predictors}, \textit{minimum samples at terminal node} and \textit{maximum depth of a tree}. Idea behind this is keep the correlation between the trees low and improve the models ability to handle noise.

\textit{Number of candidate predictors}, denoted as $K$, is one of the key hyperparameters to control the correlation between forest's trees \cite{probst2018hyperparameters}.
In cases where there are many or only few relevant predictor variables, choosing the $K$ can have high influence on the results. For example in the case of minuscule $K$ with a dataset that has only small number of important predictors most of the trees are built without the important predictor and have low prediction accuracy. \cite{bernard2009influence} Often best values for $K$ are $\sqrt(M)$ and $\log_2(M)$, where $M$ is the number of predictor variables \cite{bernard2009influence}.

Segal \cite{segal2004machine} showed that increasing the amount of noise variables lead to higher optimal terminal node size. For this reason I chose to optimize the \textit{minimum samples at terminal node}. Good default values for this hyperparameter are 1 for classification and 5 for regression \cite{probst2018hyperparameters}. Last optimized hyperparamater - \textit{maximum depth of a tree} - is related to \textit{minimum samples at terminal node} since controlling the maximum depth of the tree the algorithm might be forced to have some terminal nodes to have more samples than the minimum value requires.

\begin{table}
    \caption{Optimized hyperparameters and the tested values.}
    \begin{tabular}{ | c | c |}
    \hline
    Hyperparameter & Values\\
    \hline
    number of candidate predictors & $\sqrt(M)$, $\log_2(M)$\\
    minimum samples at terminal node & 1, 3, 5, 10, 15\\
    maximum depth of the tree & 3, 5, 8, 12, None\\
    \hline
   \end{tabular}
   \label{tab:hyperparam}
\end{table}

I used two metrics to evaluate models performance. Accuracy was used to see how many observations were classified correctly. Cross entropy loss, defined as $- \sum _ { n = 1 } ^ { N }\sum _ { c = 1 } ^ { M } y _ { nc } \log \left( p _ { nc } \right)$, where N is the number of observations, M is the number of classes, y is the binary indicator for correct class and p is the probability for that class \cite{nasrabadi2007pattern}, was used to evaluate how good models probability estimates were.

Grid search was use to test every parameter combination with a model. Cross validation of 5 folds was used. Best hyperparameter combination was chosen by looking first the highest average accuracy and then by the lowest average cross entropy loss.


\subsection{Variable importance in Random Forest}
\section{Forecasting procedure and measuments of model's forecasting accuracy}

\section{Prediction models}

\subsection{Outcome Model}
A football match can have three different outcomes: \textit{home win}, \textit{draw} or \textit{away win}. Using these three outcomes as classes random forest classifier can be used to predict the probabilities for each possible outcome. \textit{Outcome model} implements this idea and its output, the outcome probabilities, are used directly as the estimated probabilities for each outcome.
\subsection{Score Model}
\textit{Score model} is highly influenced by Groll et al.'s \cite{groll2018prediction} model that used random forest regression and Poisson distribution to simulate each match in World Cup 2018. They used random forest regression to get the expected number of goals for both of the teams. To simulate the tournament correctly they needed to have probabilities for different end results for each match. To overcome this issue they used expected number of goals from random forest regression as intensity parameter $\lambda$ in Poisson distribution $Po(\lambda)$ to draw a random number of goals for both of the teams. Both teams had their own intensity value which meant that both of the Poisson distributions were independent but conditional on the features.

Since I need the probabilities for each outcome, sampling just one possible end result for a match is not enough. For this reason, for each team, the probabilities of scoring number of goals between 0 to 10 is calculated from Poisson probability mass function. As an end result both teams have their own probabilities for scoring goals between 0 and 10 in the form of probability vector $score\_prob = \left( h _ { 1 } , h _ { 2 } , \dots , h _ { n - 1 } , h _ { n } \right)$, where N is 10. The outer product of these two score probability vectors, called the \textit{goal matrix}, has the probability estimates for each unique end result as illustrated in the figure \ref{fig:goal_matrix}. Instead of probabilities, this figure has the end result to clarify the matrix's structure.


\begin{figure}
    $\begin{bmatrix}
    0-0 & 0-1 & \cdots & 0-N \\
    1-0 & 1-1 & \cdots   &1-N \\
    \vdots & \vdots   & \ddots & \vdots \\
    N-0 & N-1 & \cdots & N-N\end{bmatrix}$
\caption{N by N Goal Matrix}
\label{fig:goal_matrix}
\end{figure}

Probabilities for match outcome are simple sums from this goal matrix, since $\sum_{i=1}\sum_{j=1}p_{ij} = 1$. Lower triangular is the probability of the home team winning, sum of diagonal is the probability of a draw and sum of upper triangular is the probability of the away team winning.

\subsection{OneVsRest model}
A multiclass classifier can be formed by combining multiple binary classifiers if outputs from binary classifiers can be merged. With random forest this is possible since each forest gives probabilities for each class. This architecture of multiple binary classifiers offers more freedom since individual classifiers can be tuned independently from the others. The idea of my \textit{OneVsRest model} is to form a multiclass classifier by combining multiple binary classifiers.

With \textit{OneVsRest model} I train single binary classifier for each outcome. A binary classifier that predicts the probability for home team's win will label all true classes (the matches where the home team won) as 1 and rest of the matches as 0. The same operation is done to predict draw and away win with correct true classes. Probability of the true class $P(c_i = 1 | x)$ is taken from each binary classifier $i$. To form the probability distribution for the match's outcome these probabilities are normalized \cite{zadrozny2002transforming}. For example probability of the home team winning is $\frac{P(c_{home_win}| x)}{P(c_{home_win}| x) + P(c_{draw}| x) + P(c_{away_win}| x)}$.

Probability estimates from binary classifier can be calibrated. This is particularly useful since some classifiers are biased with their estimates. For example, boosted trees rarely give probability estimates close to 0 or 1. This is not the case with random forest and with random forest benefiting from calibration is more problem specific. \cite{niculescu2005predicting} Two well-known calibration methods exists for binary classifiers. Platt scaling, named afterward by the inventor Platt himself \cite{platt1999probabilistic}, uses a sigmoid function to calibrate binary classifier probabilities. Probability estimates are passed through a sigmoid function $P ( c = 1 | f ) = \frac { 1 } { 1 + \exp ( A f + B ) }$ which is fitted for the problem. Here $f(x)$ is the output from the binary classifier and $A$ and $B$ are parameters that are fitted using the maximum likelihood estimation. With Platt scaling normally the assumption is that the non-calibrated probabilities tend to act like sigmoid function illustrated in the figure \ref{INSERT IMAGE}. Isotonic regression is more general calibration method. Basic assumption with isotonic regression is that function $c _ { i } = m \left( f _ { i } \right) + \epsilon _ { i }$ where $m$ is an isotonic function, is used to calibrate the probabilities. Optimal function $m$ is problem specific and learned by minimizing $\hat { m } = \arg \min _ { z } \sum \left( c _ { i } - z \left( f _ { i } \right) \right) ^ { 2 }$. \cite{zadrozny2002transforming}




\subsection{Description of our forecasting procedure}
\subsection{Forecast accuracy measurements}
Precision, recall, F-score, accuracy of probabilistic prediction: Brier score, label ranking loss
\subsection{Evaluating betting strategy}
UNIT, KELLY
% https://ac-els-cdn-com.libproxy.aalto.fi/S0169207009001708/1-s2.0-S0169207009001708-main.pdf?_tid=ad56a7bb-de9d-4230-bcde-1359866f7b5d&acdnat=1531135889_11cc5428173cb5f67c64d3e19d9f1ae3
% Stopping rule? https://search-proquest-com.libproxy.aalto.fi/docview/1986130682?pq-origsite=gscholar
