In this section, I briefly describe the main datasets:  the dataset of international football matches and the dataset of player attributes from the EA Sport's video game series FIFA. I have used both of the datasets to generate new data points. This process will be discussed in this chapter.

\section{International football match dataset}
As the primary source of data I have used results from all international football matches from November 11th, 1872 to June 6th, 2018. This dataset is provided by Kaggle \cite{matchdb} and contains match scores, tournament types, dates and more attributes that are not used in this thesis.

I don't use this dataset directly in prediction, but I use it to extract useful features. Generated features from this dataset are named as \textit{general features}.
\renewcommand{\labelitemi}{}
\begin{description}
    \itemsep0em
    \item[Elo rating:] describes the team's quality. This metric is calculated based on previous games with the following formula \begin{equation}
        \mathrm { R } _ { \mathrm { n } } = \mathrm { R } _ { \mathrm { O } } + \mathrm { K } \times \left( \mathrm { W } - \mathrm { W } _ { \mathrm { e } } \right)
    \end{equation}
    where $\mathrm { R } _ { \mathrm { n } }$ is the new Elo rating and $\mathrm { R }  _ { \mathrm { O } }$ is the old (pre-match) rating. Paramater $\mathrm { K }$ is the weight constant for the tournament played. Values for $\mathrm { K }$ are listed in the table \ref{table:weight_constant}. These values are based on the values used in the website World Football Elo rating \footnote{\url{https://www.eloratings.net/about}}. Team qualities can vary a lot within a confederation and for that reasons FIFA \footnote{\url{https://fifa.com}} has a lowering coefficient of $0.85$ ($50 \times 0.85 = 42.5$) for AFC, CONCACAF, OFC and CAF. The selected $\mathrm { K }$ for the tournament type is multiplied based on the score. It is multiplied by $1.5$ if a game is won by two goals, by $1.75$ if a game is won by three goals, and by $1 + (3/4 + (N-3)/8)$ if the game is won by four or more goals, where N is the goal difference. Parameter \mathrm { W } is the outcome of the game: $1$ for a win, $0.5$ for a draw, and $0$ for a loss. The win expectancy $\mathrm { W } _ { \mathrm { e } }$ is calculated
    \begin{equation}
        W _ { e } = 1 / \left( 10 ^ { ( - d r / 400 ) } + 1 \right)
    \end{equation}
    where $dr$ is the difference in rating.

    \item[Goal average:] describes how many goals a team has scored on average in the previous games within the timespan of four years. This metric is calculated for the home team and the away team. Feature names are \textit{home\_goal\_mean} and \textit{away\_goal\_mean}.
    \item[Goal average difference:] is the difference between the home team's \textit{goal average} and the away team's \textit{goal average}. Feature name is \textit{goal\_diff\_with\_away}.
    \item[Goal average with the opponent:] describes how many goals a team has scored on average against the opponent. The time lag is four years and the metric is calculated for both teams. Feature names are \textit{home\_goals\_with\_away} and \textit{away\_goals\_with\_home}.
\end{description}

\begin{table}[h]
        \caption{The weight constant $\mathrm { K }$ for the tournaments.}
        \label{table:weight_constant}
        \noindent
        \begin{tabular}{|p{8cm}|p{3cm}|}
            \hline
            Tournament & K \\\hline
            FIFA World Cup & 60 \\\hline
            Confederations Cup, Copa America, UEFA Euro, FIFA World Cup qualification & 50 \\\hline
            AFC Asian Cup, Gold Cup, CONCACAF Championship, Oceania Nations Cup, African Cup of Nations & $42.5$\\\hline
            African Cup of Nations qualification, AFC Asian Cup qualification, UEFA Euro qualification, CONCACAF Championship qualification, Oceania Nations Cup qualification, AFC Challenge Cup, AFC Challenge Cup qualification, Gold Cup qualification & 40 \\\hline
        \end{tabular}
    \end{table}

\section{EA Sport’s video game series FIFA's player attributes}
EA Sport's video game series FIFA describes every player in the game with several different attributes. These attributes are first collected by EA's data reviewers who are a group of coaches, professional scouts, and season ticket holders. The final value is given by EA editors based on the reviewers' answers \cite{playerattr}. From here onwards EA Sport's video game series FIFA's player attributes are called just player attributes.

Player attributes are available from August 30th, 2006 onwards \cite{sofifa}. I collected this data myself since it was not available as a single dataset.
All of the player attributes have a value in the range of 0-99. When two players are compared lower value means that the player's capability regarding that attribute is not as good as the other players. From all possible attributes, I have used 24 player attributes that were available from the beginning. These attributes are listed here with a short description that is taken from Fifplay \cite{playerattr}.

\renewcommand{\labelitemi}{}
\begin{description}
    \itemsep0em
    \item[Goalkeeper:]
    \begin{itemize}
        \itemsep0.3em
        \item[]
        \item{\textit{Diving}:} determines a player's ability to dive as a goalkeeper.
        \item{\textit{Handling}:} determines a player's ability to handle the ball and hold onto it using their hands as a goalkeeper.
        \item{\textit{Kicking}:} determines a player's ability to kick the ball as a goalkeeper.
        \item{\textit{Positioning}:} determines that how well a player is able to perform the positioning on the field as a player or on the goal line as a goalkeeper.
        \item{\textit{Reflexes}:} determines a player's ability and speed to react (reflex) for catching/saving the ball as a goalkeeper.
    \end{itemize}
    \item[Mental:]
    \begin{itemize}
        \itemsep0.3em
        \item[]
        \item{\textit{Aggression}:} determines the aggression level of a player on pushing, pulling and tackling.
        \item{\textit{Heading accuracy}:} determines a player's accuracy when using the head to pass, shoot or clear the ball.
        \item{\textit{Marking}:} determines a player's capability to mark an opposition player or players to prevent them from taking control of the ball.
    \end{itemize}
    \item[Physical:]
    \begin{itemize}
        \itemsep0.3em
        \item[]
        \item{\textit{Acceleration}:} determines the increment of a player's running speed (sprint speed) on the pitch. The acceleration rate specifies how fast a player can reach their maximum sprint speed.
        \item{\textit{Reactions}:} determines the acting speed of a player in response to the situations happening around them.
        \item{\textit{Shot Power}:} determines the strength of a player's shootings.
        \item{\textit{Sprint Speed}:} determines the speed rate of a player's sprinting (running).
        \item{\textit{Stamina}:} determines a player's ability to sustain prolonged physical or mental effort in a match.
        \item{\textit{Strength}:} determines the quality or state of being physically strong of a player.

    \end{itemize}
    \item[Skill:]
    \begin{itemize}
        \itemsep0.3em
        \item[]
        \item{\textit{Ball control}:} determines the ability of a player to control the ball on the pitch.
        \item{\textit{Crossing}:} determines the accuracy and the quality of a player's crosses.
        \item{\textit{Dribbling}:} determines a player's ability to carry the ball and past an opponent while being in control.
        \item{\textit{Finishing}:} determines the ability of a player to score (ability for finishing - How well they can finish an opportunity with a score).
        \item{\textit{Free kick accuracy}:} determines a player's accuracy for taking free kicks.
        \item{\textit{Long passing}:} determines a player's accuracy for the long and aerial passes.
        \item{\textit{Long Shots}:} determines a player's accuracy for the shots taking from long distances.
        \item{\textit{Penalties}:} determine a player's accuracy for the shots taking from the penalty kicks.
        \item{\textit{Short passing}:} determines a player's accuracy for the short passes.
        \item{\textit{Standing tackle}:} determines the ability of a player to performing standing tackle.

    \end{itemize}

\end{description}

\section{Aggregating team-level attributes}
In this section, I explain how I have combined player attributes to team-level attributes to describe a football team based on its players' capabilities.

To describe the team in general level I have used the average value from the team's 23 best players. These attributes are: \textit{overall rating, potential, age, height, weight, international reputation} and \textit{weak foot}. As an extra attribute, I calculated the average age from the top 11 players. The idea behind this attribute is to get the average age of the presumed starting lineup.

The other attributes are described by a subsection of the players. I have used my knowledge and intuition on football to select the sizes of the subsections. The goalkeeper attributes are calculated based on the team's two best values for that attributes. The average for a skill required in set piece situation, like the attributes \textit{free kick accuracy} and \textit{penalties} for example, is calculated based on the top three ratings, since in most cases a very small subset of players handle these situations. Also, \textit{sprint speed} and \textit{marking} are only calculated based on the top three values. \textit{Strength} and \textit{stamina} are calculated using the 10 best ratings. Other values are calculated using the five best ratings for each attribute.

In cases where the team doesn't have enough players to calculate the attribute's value as mentioned in the previous part I have used this formula
\begin{equation}
\frac{1}{N}\sum_{i=1}^{N}x_i * \max{(N/K, C)}.
\end{equation}
If $N$ (the number of all available ratings for the attribute), is smaller than $K$ (the required number of ratings) the average is multiplied with a coefficient that has an integer value in the inclusive range from $C$ and 1. I have set $C$ to 0.9.

\section{Historical odds}
I have collected odds for the FIFA World Cup 2018, 2014 and 2010 from Odds Portal \cite{oddsportal}. Odds Portal has a collection of odds offered by multiple betting sites. For every match, I have used the average value from the available odds. Kuypers \cite{kuypers2008} mentions that football odds are mostly fixed. Based on values that Odds Portal offer, this is not the case anymore. Many odds have changed from the initial opening odd before the match start. I have used the latest value for every odd since the data is easier to collect that way.